% One-page layout: (proof-)reading on display
%%%% \documentclass[11pt,oneside,a4paper]{book}
% Two-page layout: final printing
\documentclass[11pt,twoside,a4paper]{book}   
%=-=-=-=-=-=-=-=-=-=-=-=--=%
\usepackage[czech, english]{babel}
\usepackage[T1]{fontenc}
% pripadne pisete-li cesky, pak lze zkusit take:
% \usepackage[OT1]{fontenc} 
\usepackage[utf8]{inputenc}
\usepackage{import}
%=-=-=-=-=-=-=-=-=-=-=-=--=%
% In case of problems with PDF fonts, one may try to uncomment this line:
%\usepackage{lmodern}
%=-=-=-=-=-=-=-=-=-=-=-=--=%
%=-=-=-=-=-=-=-=-=-=-=-=--=%
% Depending on your particular TeX distribution and version of conversion tools 
% (dvips/dvipdf/ps2pdf), some (advanced | desperate) users may prefer to use 
% different settings.
% Please uncomment the following style and use your CSLaTeX (cslatex/pdfcslatex) 
% to process your work. Note however, this file is in UTF-8 and a conversion to 
% your native encoding may be required. Some settings below depend on babel 
% macros and should also be modified. See \selectlanguage \iflanguage.
%\usepackage{czech}  %%%%%\usepackage[T1]{czech} %%%%[IL2] [T1] [OT1]
%=-=-=-=-=-=-=-=-=-=-=-=--=%

%%%%%%%%%%%%%%%%%%%%%%%%%%%%%%%%%%%%%%%
% Styles required in your work follow %
%%%%%%%%%%%%%%%%%%%%%%%%%%%%%%%%%%%%%%%
\usepackage{graphicx}
%\usepackage{indentfirst} %1. odstavec jako v cestine.

\usepackage{k336_thesis_macros} % specialni makra pro formatovani DP a BP
 % muzete si vytvorit i sva vlastni v souboru k336_thesis_macros.sty
 % najdete  radu jednoduchych definic, ktere zde ani nejsou pouzity
 % napriklad: 
 % \newcommand{\bfig}{\begin{figure}\begin{center}}
 % \newcommand{\efig}{\end{center}\end{figure}}
 % umoznuje pouzit prikaz \bfig namisto \begin{figure}\begin{center} atd.



\newcommand\TypeOfWork{Master's Thesis}   \typeout{Master's Thesis} 
\newcommand\StudProgram{Open Informatics}

\newcommand\StudBranch{Artificial Intelligence}

%%%%%%%%%%%%%%%%%%%%%%%%%%%%%%%%%%%%%%%%%%%%
% Set up Work Title, Author and Supervisor
%%%%%%%%%%%%%%%%%%%%%%%%%%%%%%%%%%%%%%%%%%%%

\newcommand\WorkTitle{Sociopath: automatic local events extractor}
\newcommand\FirstandFamilyName{Galina Alperovich}
\newcommand\Supervisor{Ing. Jan Drchal, PhD}

\usepackage[
pdftitle={\WorkTitle},
pdfauthor={\FirstandFamilyName},
bookmarks=true,
colorlinks=true,
breaklinks=true,
urlcolor=red,
citecolor=blue,
linkcolor=blue,
unicode=true,
]
{hyperref}




\begin{document}

\selectlanguage{english} 


\iflanguage{czech}{
	 \typeout{************************************************}
	 \typeout{Zvoleny jazyk: cestina}
	 \typeout{Typ prace: \TypeOfWork}
	 \typeout{Studijni program: \StudProgram}
	 \typeout{Obor: \StudBranch}
	 \typeout{Jmeno: \FirstandFamilyName}
	 \typeout{Nazev prace: \WorkTitle}
	 \typeout{Vedouci prace: \Supervisor}
	 \typeout{***************************************************}
	 \newcommand\Department{Katedra počítačů}
	 \newcommand\Faculty{Fakulta elektrotechnická}
	 \newcommand\University{České vysoké učení technické v Praze}
	 \newcommand\labelSupervisor{Vedoucí práce}
	 \newcommand\labelStudProgram{Studijní program}
	 \newcommand\labelStudBranch{Obor}
}{
	 \typeout{************************************************}
	 \typeout{Language: english}
	 \typeout{Type of Work: \TypeOfWork}
	 \typeout{Study Program: \StudProgram}
	 \typeout{Study Branch: \StudBranch}
	 \typeout{Author: \FirstandFamilyName}
	 \typeout{Title: \WorkTitle}
	 \typeout{Supervisor: \Supervisor}
	 \typeout{***************************************************}
	 \newcommand\Department{Department of Computer Science and Engineering}
	 \newcommand\Faculty{Faculty of Electrical Engineering}
	 \newcommand\University{Czech Technical University in Prague}
	 \newcommand\labelSupervisor{Supervisor}
	 \newcommand\labelStudProgram{Study Program} 
	 \newcommand\labelStudBranch{Field of Study}
}



\import{Official_Task/}{official_task_description.tex}

\coverpagestarts

\acknowledgements
\noindent
I want to say thank to ... 


\declaration{Prague, June 15, 2017}

%************************************************************** 
%                       DIPLOMA THESIS TEXT 
%************************************************************** 
\import{001_Abstract/}{abstract.tex}

\tableofcontents

\listoffigures

\listoftables


%**************************************************************

\mainbodystarts
\normalfont
\parskip=0.2\baselineskip plus 0.2\baselineskip minus 0.1\baselineskip
\bibliographystyle{abbrv}
%bibliographystyle{plain}
%\bibliographystyle{psc}

%**************************************************************
% CHAPTERS
\import{01_Introduction/}{Intro1.tex}
\import{02_Chapter/}{Chap2.tex}
\import{03_Chapter/}{Chap3.tex}
\import{04_Chapter/}{Chap4.tex}
\import{05_Chapter/}{Chap5.tex}
\import{06_Chapter/}{Chap6.tex}
\import{07_Chapter/}{Chap7.tex}
\import{08_Conclusion/}{Conclusion8.tex}
\import{09_Appendix/}{Appendix_1.tex}

% BIBLIO
{
\def\CS{$\cal C\kern-0.1667em\lower.5ex\hbox{$\cal S$}\kern-0.075em $}
\bibliography{reference}
}

% CD DEscription
\import{10_CD_Description/}{CD_Descr.tex}

%**************************************************************


\end{document}
